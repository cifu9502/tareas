
%--------------------------------------------------------------------
%--------------------------------------------------------------------
% Formato para los talleres del curso de Métodos Computacionales
% Universidad de los Andes
% 2015-10
%--------------------------------------------------------------------
%--------------------------------------------------------------------

\documentclass[11pt,letterpaper]{exam}
\usepackage[utf8]{inputenc}
\usepackage[spanish]{babel}
\usepackage{graphicx}
\usepackage{mdframed}
\usepackage{tabularx}
\usepackage[absolute]{textpos} % Para poner una imagen completa en la portada
\usepackage{multirow}
\mdfdefinestyle{mystyle}{leftmargin=1cm,rightmargin=1cm,linecolor=red}
\usepackage{float}
\usepackage{hyperref}
\decimalpoint
%\usepackage{pst-barcode}
%\usepackage{auto-pst-pdf}

\newcommand{\base}[1]{\underline{\hspace{#1}}}
\boxedpoints
\pointname{ pt}
%\extrawidth{0.75in}
%\extrafootheight{-0.5in}
\extraheadheight{-0.15in}
%\pagestyle{head}

%\noprintanswers
%\printanswers
\renewcommand{\solutiontitle}{}
\SolutionEmphasis{\color{blue}}

\usepackage{upquote,textcomp}
\newcommand\upquote[1]{\textquotesingle#1\textquotesingle} % To fix straight quotes in verbatim

\begin{document}
\begin{center}
{\Large Métodos Computacionales} \\
Taller 4 - \textsc{Interpolación, Álgebra Lineal y Fourier} \\
Profesor: Sebastián Pérez Saaibi\\
Fecha de Publicación: {\small \it Marzo 6 de 2015}\\
\end{center}

\begin{textblock*}{40mm}(10mm,20mm)
  \includegraphics[width=3cm]{logoUniandes.png}
\end{textblock*}

\begin{textblock*}{40mm}(161mm,20mm)
  \includegraphics[width=3cm]{logoUniandes.png}
\end{textblock*}

\vspace{0.5cm}

{\Large Fecha de Entrega:  \bf Marzo 19 de 2015 antes de las 21:59 COT}

\vspace{0.5cm}

{\Large Instrucciones de Entrega}\\


Todo el código fuente y los datos se debe encontrar en un repositorio público en github con un commit final hecho antes de la fecha de entrega. El nombre del repositorio debe ser \newline \verb+CM20151_HW4_NombreApellido+, por ejemplo yo debo crear un repositorio llamado \newline \verb+CM20151_HW4_SebastianPerez+. El link al repositorio lo deben enviar a través de \textbf{sicuaplus} antes de la fecha/hora límite.

En cada parte del ejercicio se entrega 1/3 de los puntos si el código propuesto es razonable, 1/3 si se puede ejecutar y 1/3 si entrega resultados correctos.

\vspace{0.5cm}


\begin{questions}

\question[35] {\bf Calentamiento Global?}

\begin{parts}
\part[15] Cree un script de R que descargue los datos históricos de temperatura para 5 ciudades de Colombia: Bogotá, Cali, Bucaramanga, Barranquilla e Ipiales \footnote{Puese usar los datos de \url{http://data.giss.nasa.gov/gistemp/station_data/} para ese propósito}. El script debe generar como salida una tabla en representación limpia \footnote{Tidy Data: \url{vita.had.co.nz/papers/tidy-data.pdf}} llamada \verb+temperaturas.csv+ con las siguientes columnnas: \verb+año,mes,fecha,ciudad,temperatura+. Asimismo, debe generar una gráfica en formato \verb+.png+ de calidad de artículo que muestre las series de tiempo para las 5 ciudades. 
\part[13] Cree un script de python llamado \verb+interp_temperaturas.py+ que haga y grafique una interpolación lineal, polinómica y por splines (escoja polinomio y órden de los splines) para cada una de las ciudades anteriores. Cuál método es mejor? Justifique estadísticamente su escogencia.
\part[7] Cree un html (ya sea usando un cuaderno \verb+.Rmd+ o \verb+ipython+) en el que establezca y justifique estadísticamente su posición con respecto al calentamiento global en referencia a los datos de las 5 ciudades anteriores. 

\end{parts}

\newpage


\question[30] {\bf Estudiando Rios y Oceanos} En este ejercicio, queremos ver el impacto de variables atmosféricas y términas en el comportamiento de rios y oceanos.

\begin{parts}
\part[10] A partir de este catálogo de descargas hidrográficas \footnote{\url{http://www.cgd.ucar.edu/cas/catalog/surface/dai-runoff/coastal-stns-byVol-updated-oct2007.txt}}, cree un script de python llamado \verb+top_300_rios.py+ que genere un archivo llamado \verb+top_300_rios.csv+ que solo contenga los con mayor tasa de flujo.
\part[10] Usando el módulo \verb+Basemap+, cree un script de python que genere una gráfica en formato \verb+.png+ con la tasa de flujo de los principales 150 rios sobre un mapa.
\part[10] , cree un script de python llamado \verb+interp_mapa.py+ que grafique un archivo .nc con las temperaturas medias del aire \footnote{\url{ftp://ftp.cdc.noaa.gov/Datasets/ncep.reanalysis.derived/surface/air.mon.ltm.nc}}. Haga una interpolación \emph{nearest neighbors} de esos valores, genere un gráfico y comente.
\end{parts}


\question[35] {\bf El poder de su voz}

\begin{parts}
\part[5] Escriba un script en \verb+C+ llamado \verb+grabar_mi_nombre.C+ que grabe un archivo \verb+.wav+ en el que diga su nombre completo (con dos apellidos).
\part[10] Escriba un script llamado \verb+grafica_mi_voz.py+ que lea este \verb+.wav+ y grafique en un \verb+mi_voz.png+ la señal modulada de su voz. Tenga en cuenta las leyendas, nombres de los ejes y estética del gráfico.
\part[10] Escriba un script llamado \verb+fft_de_mi_voz.py+ que ejecute una transformada rápida de fourier para su nombre, y encuentre el armónico más grande. Este script debe generar una imagen llamada \verb+mivoz_fft.png+ Describa el método utilizado para encontrar dicho armónico.
\part[10] Haga un Makefile que genere las salidas descritas anteriormente en el órden adecuado y genere todas las salidas.
\end{parts}


\end{questions}
\end{document}
